%%%%%%%%%%%%%%%%%%%%%%%%%%%%%%%%%%%%%%%%%%%%%%%%%%%%%%%%%%%%%%%%%%%%%%%%%%%%%%%%
%2345678901234567890123456789012345678901234567890123456789012345678901234567890
%        1         2         3         4         5         6         7         8

\documentclass[letterpaper, 10 pt, conference]{ieeeconf}  % Comment this line out if you need a4paper

%\documentclass[a4paper, 10pt, conference]{ieeeconf}      % Use this line for a4 paper

\IEEEoverridecommandlockouts                              % This command is only needed if 
                                                          % you want to use the \thanks command

\overrideIEEEmargins                                      % Needed to meet printer requirements.

% See the \addtolength command later in the file to balance the column lengths
% on the last page of the document

% The following packages can be found on http:\\www.ctan.org
%\usepackage{graphics} % for pdf, bitmapped graphics files
%\usepackage{epsfig} % for postscript graphics files
%\usepackage{mathptmx} % assumes new font selection scheme installed
%\usepackage{times} % assumes new font selection scheme installed
\usepackage{amsmath} % assumes amsmath package installed
\usepackage{amssymb}  % assumes amsmath package installed

%\DeclareMathOperator*{\argmin}{arg\,min}
%\DeclareMathOperator*{\argmax}{arg\,max}

\title{\LARGE \bf
Action Selection for Model Disambiguation
}


\author{P. Michael Furlong
\thanks{*This work was supported by LAKELANDER}% <-this % stops a space
\thanks{$^{1}$P. Michael Furlong is with NASA Ames Intelligent Robotics Group
        NASA Ames Research Center, Moffatt Field, CA 94035, USA
        {\tt\small padraig.m.furlong@nasa.gov}}%
}


\begin{document}



\maketitle
\thispagestyle{empty}
\pagestyle{empty}


%%%%%%%%%%%%%%%%%%%%%%%%%%%%%%%%%%%%%%%%%%%%%%%%%%%%%%%%%%%%%%%%%%%%%%%%%%%%%%%%
\begin{abstract}

\textbf{TODO}

\end{abstract}


%%%%%%%%%%%%%%%%%%%%%%%%%%%%%%%%%%%%%%%%%%%%%%%%%%%%%%%%%%%%%%%%%%%%%%%%%%%%%%%%
\section{INTRODUCTION}

\begin{enumerate}
\item Learning algorithms need to choose between models that represent the data
\item While learning it is important to economically spend sampling resources to disambiguate which models
\end{enumerate}

\section{BACKGROUND}

\begin{enumerate}
\item $A^2$ learner finds points where classifiers disagree.
\item UCB Improved uses bandit arms where their expected value overlaps
\item Maximum entropy sampling
\item Lindly 1956 defined a measure of how much a sample improves the certainty. 
\end{enumerate}

\section{MODEL SELECTION}

\begin{enumerate}
\item Bayesian Information Criterion
\item Principled approach to actions that are most informative about the underlying model.
\end{enumerate}

\begin{equation} \label{eq1}
x = \arg\max_{x' \in S} H\left(K|\hat{f}\left(x'\right)\right) - H\left(K\right) 
\end{equation}

Where $K$ is the kernel and $x'$ is a point in the support of the function to be learned, $f\left(x\right)$

\subsection{Equations}

The equations are an exception to the prescribed specifications of this template. You will need to determine whether or not your equation should be typed using either the Times New Roman or the Symbol font (please no other font). To create multileveled equations, it may be necessary to treat the equation as a graphic and insert it into the text after your paper is styled. Number equations consecutively. Equation numbers, within parentheses, are to position flush right, as in (1), using a right tab stop. To make your equations more compact, you may use the solidus ( / ), the exp function, or appropriate exponents. Italicize Roman symbols for quantities and variables, but not Greek symbols. Use a long dash rather than a hyphen for a minus sign. Punctuate equations with commas or periods when they are part of a sentence, as in

$$
\alpha + \beta = \chi \eqno{(1)}
$$

\section{EXPERIMENTS}

\begin{enumerate}
\item Restrict range of functions to $\left[0,1\right]$ (wlog)
\item Objective measure: Reconstruction error over the range finely sampled after the fact.
\item Initially start with a uniform prior over the models
\item With and without measurement noise.
\item Using Gaussian process regression \cite{bishop2006pattern} to represent the model selection 
\item Control algorithms: Random and uniform sampling, max entropy sampling over 
\item Stationary functions
\item Nonstationary functions
\end{enumerate}

\section{RESULTS}

plots
\begin{enumerate}
\item Reconstruction error vs number of samples for different strategies for stationary functions
\item Reconstruction error vs number of samples for different strategies for nonstationary functions
\item Time to compute action (on average) -- function of number of data points collected
\end{enumerate}


\section{CONCLUSIONS}

A conclusion section is not required. Although a conclusion may review the main points of the paper, do not replicate the abstract as the conclusion. A conclusion might elaborate on the importance of the work or suggest applications and extensions. 

\addtolength{\textheight}{-12cm}   % This command serves to balance the column lengths
                                  % on the last page of the document manually. It shortens
                                  % the textheight of the last page by a suitable amount.
                                  % This command does not take effect until the next page
                                  % so it should come on the page before the last. Make
                                  % sure that you do not shorten the textheight too much.

%%%%%%%%%%%%%%%%%%%%%%%%%%%%%%%%%%%%%%%%%%%%%%%%%%%%%%%%%%%%%%%%%%%%%%%%%%%%%%%%



%%%%%%%%%%%%%%%%%%%%%%%%%%%%%%%%%%%%%%%%%%%%%%%%%%%%%%%%%%%%%%%%%%%%%%%%%%%%%%%%



%%%%%%%%%%%%%%%%%%%%%%%%%%%%%%%%%%%%%%%%%%%%%%%%%%%%%%%%%%%%%%%%%%%%%%%%%%%%%%%%
\section*{APPENDIX}

Appendixes should appear before the acknowledgment.

\section*{ACKNOWLEDGMENT}

The preferred spelling of the word �acknowledgment� in America is without an �e� after the �g�. Avoid the stilted expression, �One of us (R. B. G.) thanks . . .�  Instead, try �R. B. G. thanks�. Put sponsor acknowledgments in the unnumbered footnote on the first page.



%%%%%%%%%%%%%%%%%%%%%%%%%%%%%%%%%%%%%%%%%%%%%%%%%%%%%%%%%%%%%%%%%%%%%%%%%%%%%%%%

References are important to the reader; therefore, each citation must be complete and correct. If at all possible, references should be commonly available publications.


\bibliographystyle{ieeetran}
\bibliography{references}

\end{document}
